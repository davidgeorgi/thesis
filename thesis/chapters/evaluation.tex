In this Chapter, the performance of the text-aware process prediction model is evaluated based on real-life event data.
First, the data sets and evaluation method are described. Then, the performance of differently parameterized models is evaluated and analyzed in-depth.


\section{Datasets}

\begin{table}[!htbp]
	\begin{tabularx}{\textwidth}{l l l }
		\toprule
		\textbf{Dataset} & \textbf{D1} & \textbf{D2}  \\
		\midrule
		Number of traces & & \\
		Number of trace variants & & \\
		Number of events & & \\
		Events per trace (avg.) & & \\
		Median case duration (days) & & \\
		Mean case duration (days)& & \\
		Number of activities & & \\
		Vocabulary size & & \\
		\bottomrule
	\end{tabularx}
	\caption[Evaluated datasets]{Datasets used for the evaluation of the text-aware process prediction model.}
	\label{tab:packages}
\end{table}

\section{Evaluation Method}

Each event log is evaluated in a consistent procedure.
In the first step, the event log is separated into a training and test log. 
The training log consists of the first 2/3 chronologically ordered traces and is used to fit the prediction model to the historical data.
The remaining 1/3 of traces are used to measure the prediction performance.
For each trace $\sigma$ in the test log, all prefixes $hd^k(\sigma)$ of length $1 \leq k \leq |\sigma| - 1$ are considered as instances for prediction.
During the training of the LSTM model, 20\% of the training log is used for validation, i.e. the training is stopped, if the error on the validation log is not decreasing anymore, in order to avoid overfitting.

For classification (i.e. categorical prediction) task, like next event and outcome prediction, the accuracy and Brier score \cite{Brier1950VERIFICATIONOF} are utilized as metric.
The accuracy is computed as the number of correct predictions $t$ divided by the total number of predictions $n$, i.e. 
\begin{equation*}
	\textrm{accuracy} = \dfrac{t}{n} = \dfrac{\textrm{\# correct predictions}}{\textrm{\# total predictions}}.
\end{equation*}
The Brier score uses the softmax output vector  $\vec{\hat{y}}$ of the prediction model and is calculated as the mean squared error of the predicted likelihoods over each possible outcome, precisely
\begin{equation*}
	\textrm{Brier score} = \dfrac{1}{n} \sum_{i=1}^{n}\sum_{j=1}^{p}(\hat{y}_{ij}-y_{ij})^2,
\end{equation*}
where $n$ is total number of prediction instances, $p$ is the number of possible outcomes and $\vec{y}$ is the vector indicating the true outcome.
The Brier score also considers the confidence of the prediction model, unlike the accuracy score, which only considers the final prediction choice of the model .
An accuracy value of 1 and a Brier score of 0 are the most desirable scores.

For regression tasks, like the next event time and the case cycle time predictions, the mean absolute error (MAE) is computed to measure the prediction performance. The mean absolute error indicates the average absolute difference between the predicted value $\hat{y}$ and the true value $y$,  i.e.
\begin{equation*}
	\textrm{MAE} = \dfrac{1}{n}\sum_{i=1}^{n}|\hat{y_i} - y_i|.
\end{equation*}
This error metric gives a more intuitive interpretation and is less sensitive to outliers compared to similar metrics like the mean squared error (MSE).

\section{Next Event Prediction}

\begin{table}[!htbp]
	\begin{tabularx}{\textwidth}{l l l l }
		\toprule
		& & &  \\
		\midrule
		& & & \\
		\bottomrule
	\end{tabularx}
	\caption[Experimental results for the next event prediction]{Experimental results for the next event prediction.}
	\label{tab:next-event}
\end{table}

\section{Case Cycle Time Prediction}

\begin{table}[!htbp]
	\begin{tabularx}{\textwidth}{l l l l }
		\toprule
		& & &  \\
		\midrule
		& & & \\
		\bottomrule
	\end{tabularx}
	\caption[Experimental results for the case cycle time prediction]{Experimental results for the case cycle time prediction.}
	\label{tab:case-cycle-time}
\end{table}

\section{Outcome Prediction}

\begin{table}[!htbp]
	\begin{tabularx}{\textwidth}{l l l l }
		\toprule
		& & &  \\
		\midrule
		& & & \\
		\bottomrule
	\end{tabularx}
	\caption[Experimental results for the case outcome prediction]{Experimental results for the case outcome prediction.}
	\label{tab:outcome}
\end{table}

\section{Future Path Prediction}

\begin{table}[!htbp]
	\begin{tabularx}{\textwidth}{l l l l }
		\toprule
		& & &  \\
		\midrule
		& & & \\
		\bottomrule
	\end{tabularx}
	\caption[Experimental results for the future path prediction]{Experimental results for the future path prediction.}
	\label{tab:future-path}
\end{table}