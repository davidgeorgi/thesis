


\section{Motivation}

The rapid growth of data generated by large-scale information systems leads to new opportunities for  society and businesses.
In order to turn the massive amount of data into value, efficient techniques are needed, that are able to extract new information.
The generated value can come in different forms: Depending on the context, visualizations, models, aggregated data or predictions can be of interest. 

A remarkable subset of this data is described as \textit{event data}, which is generated by software systems, that execute processes and save the executed activities in logs.
With the non stopping rise of digitization and processes switching from the analog world to the digital one, the theoretical value of this data is exploding.

This motivates scientist and business to take advantage of this potential.
The scientific engagement in this discipline is described by \textit{process mining}, which bridges the gab between the data-driven characteristic of data science and the process centric view of process science \cite{DBLP:books/sp/Aalst16}.
The ongoing success of progress mining in research has been transferred to business that successfully offer or utilize this technology.
Celonis, which can be considered as the biggest company in this field, has been valued 2.5 billion dollar only 9 years after the company was founded.



\section{Problem Statement}

\section{Research Goals}

\section{Contribution}

\section{Thesis Structure}